
\documentclass[overlapped,line,letterpaper]{res}

\usepackage{ifpdf}

\ifpdf
  \usepackage[pdftex]{hyperref}
\else
  \usepackage[hypertex]{hyperref}
\fi



%%===========================================================================%%

\begin{document}

%---------------------------------------------------------------------------
% Document Specific Customizations

% Make lists without bullets and with no indentation
\setlength{\leftmargini}{0em}
\renewcommand{\labelitemi}{}

% Use large bold font for printed name at top of pages
\renewcommand{\namefont}{\large\textbf}

%---------------------------------------------------------------------------

\name{James A. Stickney}

\begin{resume}

\begin{ncolumn}{2}

41 Pleasant Steet & Phone:  (774) 253-1786 \\
Wakefield               & Email: {\tt jim.stickney@gmail.com} \\
Massachusetts 01880, USA
\end{ncolumn}




%---------------------------------------------------------------------------

\section{ \bf Biography }

Dr. Stickney has been researching cold atom based sensors since 2001.  His principal contributions have been in the areas of Cold Atom Interferometry using both Bose Einstein Condensation (BEC) and thermal atomic clouds for the application of inertial navigation. He was also a pioneer in the field of Atomtronics where he was one of the first to develop a theoretical model for a Cold Atom Transistor.  

After graduate school, Dr. Stickney provided theoretical support to the AFRL Cold Atom Optics Effort as a Contractor with Stuart Radiance Laboratory (SRL) in Bedford, MA. SRL is part of the Space Dynamics Laboratory of Utah State University. While at SRL he was the Principal Investigator overseeing a \$4.8 M Research and Development Contract between Utah State University and The Air Force Research Laboratory. 

Dr. Stickney then became a government employee at the Air Force Research Laboratory’s (AFRL) Battlespace Surveillance Innovation Center located at Hanscom Air Force Base in Bedford, Massachusetts. There he was a member of the Cold Atom Optics Effort, where he provided theoretical support for the experimental effort. He also developed a modeling tool to aid in the design of cold atom experiments. He has assisted in the design and building of several, novel, cold atom sensors which could potentially revolutionize cold atom navigation. 

Since the move of the Cold Atom Optics Effort to Kirtland AFB, NM, Dr. Stickney has returned to his work with the Stuart Radiance Laboratory (SRL) in Bedford, MA.  Under contract with Space Dynamics Laboratory in of Utah State University, he was able to continue to support the AFRL Cold Atom Optics Effort remotely.

Dr. Stickney has extensive experience in both analytic and numerical modeling of quantum mechanical, electro-magnetic, optical, and non-equilibrium statistical mechanical systems.  He has expertise in programing in C, python, and matlab.  He has experience developing 3D visualizations of complex data using VTK and had developed codes that utilize general purpose computing on graphics processors, specifically CUDA, as well as parallel programing using the message passing interface (MPI). 

%---------------------------------------------------------------------------


\section{\bf Education}
Ph.D. Physics, Worcester Polytechnic Institute, May 2008 

M.S. Physics, Worcester Polytechnic Institute, 2003

%---------------------------------------------------------------------------

\section{\bf Research Experience}

\begin{format}
\title{l}\dates{r}\\
\employer{l}\location{r}\\
\body\\
\end{format}

\title{Physicist}
\employer{Space Dynamics Laboratory}
\location{Bedford, MA}
\dates{2011-present}
\begin{position}
Designed several novel atom chips for use in cold atom interferometery.  These include a new type  of chip that is capable of creating magnetic traps that are tune-able over a very large range of trap shapes.  Analyzed the effects of trap shape on trapped atom interferometery.  Developed numerical models of atom traps and the dynamics of atomic clouds in traps that utilize general purpose programming on graphics processors.  These new models have increased the real world performance of the software by greater than 40 times.  
\end{position}

\title{Physicist}
\employer{United States Air Force}
\location{Hanscom AFB, MA}
\dates{2009-2011}
\begin{position}
Theoretical and experimental work on atom interferometry for inertial navigation systems at the Air Force Research
Laboratory.  Continued development of cold atom modeling tool.  Developed a new process for building atom chips using direct bonded copper substrates.  Analyzed the
optimal shape of wires used on atom chips.  
 \end{position}



\title{Research Scientist}
\employer{Space Dynamics Laboratory}
\location{Bedford, MA}
\dates{2007-2009}
\begin{position}
 Theoretical and experimental work on atom interferometry for inertial navigation systems.  Introduced a new type of
 interferometer design for use in rotation sensing and analyzed the performance limits of this sensor.  Developed a new
 type of atom trap that produce a very harmonic potential.  This type of trap will be necessary for the realization
 of a trapped atom interferometer.  Wrote a cold atom modeling tool to speed the development of cold atom based
 sensors.  This tool contains a trap optimizer and a Monte-Carlo solver for calculating the temperature, density,
 location, and number of trapped atoms.  
\end{position}


\title{Graduate Research Assistant}
\employer{Alex A. Zozulya}
\location{Worcester Polytechnic Institute}
\dates{2001-2007}
\begin{position}
 Theoretical work on Bose Einstein condensate(BEC) based devices.  The operation of BEC based interferometers that use a double well potential was
 extensively analyzed.  A recombination instability for this type of interferometer was predicted.  A simple BEC
 transistor was proposed that uses a triple well potential.  Analysis of BEC based interferometers that use optical
 pulses to manipulate the atoms was preformed.  A very simple way to understand why this type of interferometer loses
 its coherence was presented as well as methods for increasing the coherence.  
\end{position}


\title{Undergraduate Research Assistant}
\employer{Grover A Swartzlander}
\location{Worcester Polytechnic Institute}
\dates{1998-2000}
\begin{position}
  Experimental work building optical tweezers for trapping and manipulation micron sized objects. 
\end{position}







\section{\bf Peer-Reviewed Scientific Publications}

[1] ``Expansion of a {B}ose-{E}instein condensate from a micro trap into a waveguide,'' J. A. Stickney and A.A. Zozulya,
 Phys. Rev. A. \textbf{65}, 053612 (2002).

[2] ``Wave-function recombination instability in cold-atom interferometers,'' J. A. Stickney and A. A. Zozulya,
Phys. Rev. A. \textbf{66}, 053601 (2002).

[3] ``Influence of nonadiabaticity and nonlinearity on the operation of cold-atom beam splitters,'' J. A. Stickney and A. A. Zozulya,
Phys. Rev. A. \textbf{68}, 013611 (2003).

[4] ``Decrease in the visibility of the interference fringes in a cold-atom accelerometer,'' J. A. Stickney and A. A. Zozulya,
Phys. Rev. A. \textbf{69}, 063611 (2004).

[5] ``Transistor like behavior of a {B}ose-{E}instein condensate in a triple-well potential,'' J. A. Stickney, D. Z. Anderson, and A. A. Zozulya,
Phys. Rev. A. \textbf{75}, 013608 (2007).

[6] ``Increasing the coherence time of {B}ose-{E}instein-condensate interferometers with optical control of dynamics,'' J. A. Stickney, D. Z. Anderson, and A. A. Zozulya, Phys. Rev. A \textbf{75}, 063603 (2007).

[7] ``MATLAB codes for teaching quantum physics: Part 1,'' R. Garcia, A. Zozulya, J. Stickney, arXiv:0704.1622 (2007).

[8] ``Theoretical analysis of a single- and double-reflection atom interferometer in a weakly confining magnetic trap,''
J. A. Stickney, R. P. Kafle, D. Z. Anderson, and A. A. Zozulya, Phys. Rev. A \textbf{77}, 043604 (2008).

[9] ``Collisional decoherence in trapped-atom interferometers that use nondegenerate sources,'' J. A. Stickney,
M. B. Squires, J. Scoville, P. Baker, and S. Miller, Phys. Rev. A \textbf{79}, 013618 (2009).

[10] ``Adjustable microchip ring trap for cold atoms and molecules,''
P.M. Baker, J.A. Stickney, M.B. Squires, J.A. Scoville, E.J. Carlson, W.R. Buchwald, S.M. Miller,
Phys. Rev. A. \textbf{80}, 063615 (2009).

[11] ``Atom chips on direct bonded copper substrates,'' Matthew B. Squires, James A. Stickney, Evan J. Carlson, Paul M. Baker, Walter R. Buchwald, Sandra Wentzell, and Steven M. Miller,
Rev. Sci. Instrum \textbf{82}, 023101 (2011).

%[11] ``Atom chips on direct bonded copper substrates,'' M.B. Squires, J.A. Stickney, E.J. Carlson, P.M. Baker,
%W.R. Buchwald, S.M. Miller,         
%------------------------------------------------------------------------------

\section{\bf Selected Posters and Presentations}

[1] ``Transistor like behavior of a Bose Einstein condensate in triple well potential,'' March Meeting of the American Physics Society, Baltimore, MD, March 2006.

[2] ``BEC transistor,'' DAMOP Meeting 2006, Knoxville, TN, May 2006.

[3] ``Analysis of the coherence time of a Bose-Einstein-condensate interferometer with optical control of dynamics,''
March Meeting of the American Physics Society, Denver, CO, March 2007.

[4] ``Atomoptics and Atomtronics'' Physics Colloquia, WPI, Worcester MA, April 2007.

[5] ``Theoretical analysis of trapped atom interferometers using laser cooled sources,'' International Conference on
Atomic Physics, Storrs, CT, July 2008.

[5] ``Collisional decoherence in trapped atom interferometers that use non-degenerate sources,'' gBECi Review, Las
Vegas, NV, December 2009.


%---------------------------------------------------------------------------



%---------------------------------------------------------------------------




\section{\bf Skills:} 

\begin{itemize}

\item Experienced in computational modeling, including:
\begin{itemize} 
\item Monte-Carlo methods
\item parallel programing (MPI)
\item graphical user interfaces
\item 3D visualizations of complex data (VTK)
\item Experience in developing embedded systems for control systems.  
\item General purpose programming on graphics processors
\end{itemize}

\item Programing Languages:
\begin{itemize}
\item Matlab
\item Python
\item Fortran
\item C, C++
\item CUDA
\end{itemize}

\item Expertise using software packages including:
\begin{itemize}
\item \LaTeX
\item Blender
\item Emacs
\item Linux
\item Microsoft windows
\item Microsoft office 
\end{itemize}

\end{itemize}

%%---------------------------------------------------------------------------%%


\section{\bf References}

\noindent
Matthew B. Squires \\
Senior Physicist AFRL/RVBYE, \\
Bldg 570, Rm 1154.
3550 Aberdeen Ave SE.
Kirtland AFB NM 87117
 \\
TEL: (505) 853-2739.  EMAIL: matthew.squires@us.af.mil.

\noindent
Alex A. Zozulya \\
Department of Physics, Worcester Polytechnic Institute, \\
100 Institute Rd., 213C Olin Hall, Worcester, MA 01609.  \\
TEL:  (508)831-5254.  FAX: (508) 831-5886.  EMAIL: zozulya@wpi.edu.

\noindent
Patricia H. Doherty \\
Institute for Scientific Research, Boston College\\
140 Commonwealth Avenue. Chestnut Hill, MA 02467 \\
TEL: (671) 552-8767.  FAX: (617) 552-2818.  EMAIL: Patrica.Doherty@bc.edu


 


\end{resume}

\end{document}

%%===========================================================================%%
